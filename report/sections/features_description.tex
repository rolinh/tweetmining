\section{Features description}

In this section, we first describe the features that are relevant for our 
work, and then we discuss about their usefulness in the classification process. 
We finally mention the other features that have been removed from the final 
experiment. The table \ref{tab:features} lists all the features.

\begin{table}[!h]
 \centering
 \begin{tabular}{|c|l|}
  \hline
  \tabhead{\#} &
  \multicolumn{1}{|p{0.7\columnwidth}|}{\centering\tabhead{Features used}} \\
  \hline
  1  & is the tweet a retweet \\
  2  & is the tweet a reply \\
  3  & number of followers \\
  4  & number of times "favorited" \\
  5  & number of tweets issued by the author \\
  6  & is the author a verfied account \\
  7  & number of user's friends \\
  8  & number of hashtags \\
  9  & number of words in the tweet \\
  10 & number of users mentionned \\
  11 & has an URL \\
  12 & TF feature \\
  13 & TF-IDF feature \\
  14 & tweet's age \\
  \hline
 \end{tabular}
 \caption{This table lists the features used in our work. Their order has no 
  specific meaning.}
 \label{tab:features}
\end{table}

The first feature indicates if a tweet is a retweet or not. This information is 
extracted from the text of the tweet which appears as a citation of the 
original tweet with the form "RT @username:". 

The feature number 2 determines whether a tweet is a reply to another 
tweet.This information is provided by the attribute "in\_reply\_to\_user\_id" 
which contains a non-null value if the tweet is a reply.

The third feature counts the number of followers the author has. This value 
is given by the \emph{User} object attached to a tweet.

The feature number 4 indicates how many times a tweet has been "favorited" by 
other users. This count is given by the attribute "favorite\_count"  of a tweet 
object.

The fifth feature gives the number of other tweets (including retweets) issued 
by the author. This information is provided by the attribute "statuses\_count" 
of the \emph{User} object.

The feature number six indicates if the author of the tweet has a verified 
account. Such accounts establish the authenticity of the user's identity. In 
our case, this means that the author of the tweet is famous and then he will be 
probably retweeted.
